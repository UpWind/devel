\documentclass[12pt]{article}

\usepackage{hyperref}
\usepackage{examplep}

\title{How to set up the environment for the project}

\begin{document}
\maketitle

\section{Set up the compiler}

	This procedure is only needed for Windows environment in case Qt creator does not find the header files in the project.

	The project can be then compiled using the command line.
	For this, two paths must be added to the \%PATH\% system variable.
	The window to change it can be accesed right clicking the computer item in the start menu, clicking advanced system settings, and clicking Environment variables.

	Then edit the path variable by adding
	\begin{verbatim};<QtSDK\>\mingw\bin;<QtSDK\>\Desktop\Qt\4.7.4\mingw\bin\end{verbatim} to the end of the text, where \textless{}QtSDK\textgreater{} is the path where QtSDK was installed.

	After this open a terminal, navigate to the root of the project and type \texttt{qmake \&\& make}.

\section{Set up the database}
\begin{itemize}
	\item Install PostgreSQL 8.3 or 8.4 from the official page: \url{http://www.postgresql.org/download/}
	\item Install PostGIS 1.5 from the official site: \url{http://postgis.refractions.net/download/}
	\item Install the postgresql cube module.
	\item Create a new database with pgAdmin III
	\item Right-click the database icon it, click restore and navigate where the backup file is, it should be in te repository, and press ok.
\end{itemize}
If there are any errors on OS X or Linux because liblwgeom can't be found make it a link to the postgis library, for example: \begin{verbatim}sudo ln -s /usr/lib/postgis/1.5.1/postgres/8.4/lib/postgis-1.5.so /usr/lib/postgresql/8.4/lib/liblwgeom\end{verbatim}. Other errors that were found when importing in all platforms, like missing ``\$libdir/cube'', didn't seem to affect the database, so it's recommended to check if the tables are full just in case the database is allright.

\end{document}
